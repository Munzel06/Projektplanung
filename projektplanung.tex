
\chapter{Projektplanung}

\section{Projektorganisation}
\begin{center}
\begin{table}[h]
\begin{tabularx}{\textwidth}{X|X}
    \textbf{Person} & \textbf{Funktion} \\ \hline
    Andrea Bardill & Projektleiter \\
    Michael Brand & Auftraggeber der Schule \\
    5 Eltern \& 5 Teilnehmer*innen & Teilnehmer*innen der Stichprobe \\
    3 Personen ohne Pfadierfahrung & Teilnehmer*innen der Testung
\end{tabularx}
\caption{Projektorganisation}
\end{table}
\end{center}

\section{Projektstruktur und Zeitplan}
Um das ganze Projekt zu strukturieren wurde die Mehtode der Userstories gewählt \cite{userstories}. Userstories erlauben es alles was man ereichen möchte in kleine Aufgaben herunter zu brechen, um diese dann schön systematisch abzuarbeiten. Ein weiterer Vorteil dieser Userstories ist es, dass man wenn man zu wenig Zeit hat um alle zu beenden, denoch ein tolles Produkt hat da das wichtigste am Anfang erledigt wird. \par Den Aufgaben werden dann im Anschluss je eine Zeit zugeteilt, die es etwa benötigt um die Aufgabe zu bewältigen. Aus der Zusammenrechnung all dieser Zeiten ging hervor, dass dieses Projekt circa 150 Stunden in anspruch nehmen wird. Die Userstories wurden in einem externen Dokument abgegeben worden.
\newpage
\section{Ressourcen und Finanzierung}

\begin{center}
\begin{table}[h]
\begin{tabularx}{\textwidth}{X|X|X}
    \textbf{Ressource} & \textbf{Betrag für Ressource (Fr.)} & \textbf{Finanzierung} \\ \hline
    Laptop & Vorhanden & - \\
    Android Studio & 0 & - \\
    \LaTeX -Editor & 0 & - \\
    Literatur & 20 & Privat\\
    Aufladen im Google Play Store & 25 & Privat \\
    Unerwartetes & 50 & Privat \\ \hline
    \textbf{Total} & 95 Fr. & 
\end{tabularx}
\caption{Ressourcen und Finanzierung}
\end{table}
\end{center}

\section{Projektumfeldanalyse}
\begin{table}[h]
\begin{tabularx}{\textwidth}{p{0.2\textwidth}|X|X}
    \textbf{Einflussfaktor} & \textbf{Auswirkungen/Folgen} & \textbf{Massnahmen} \\ \hline
    Laptop geht kaputt & Daten gehen verloren, Projekt kann nicht rechtzeitig fertiggestellt werden & Die Dateien der App sowie der Berichte werden auf einer Cloud gespeichert und es werden regelmässige Back-ups gemacht\\ \hline
    Fehlende Versionskontrolle & Ein Programmieransatz ist nicht machbar oder beeinträchtigt den Rest der App und es ist nicht möglich auf eine alter Version der App zuzugreifen & Die Appdateien werden regelmässig auf GitHub geladen. Dies ermöglicht Versionskontrolle. \\ \hline
    Zu wenige Personen für die Stichprobe und den Testlauf & Ungenaue Informationen aus dem Testlauf und der Stichprobe & Früh damit beginnen Personen zu sammeln \\ \hline
    Personen des Testlaufs trainieren nicht mit der App & Die Personen haben nicht das erwartete wissen, da zu wenig gelernt wurde & Die Testpersonen müssen aufschreiben wieviel sie mit der App gelernt haben \\
\end{tabularx}
\caption{Projektumfeldanalyse}
\end{table}