\setcounter{chapter}{4}
\chapter{Controlling}
\section{Erstes Zwischen-Controlling: Projektantrag und Projektplanung (07.04.2024)}

Das Projekt ist, bis jetzt, wie erwartet verlaufen. Die Ausnahme bilden die Stunden, welche investiert wurden, um Java und Android Studio zu lernen. Hierbei war gerade anfangs die Motivation voll da. Im verlauf des Projekts wurde erkannt, dass es noch zu früh ist um wirklich das Programmieren zu erlernen. Als folge wurden die Zeiten des Programmierens reduziert. Kleinere Abweichungen gab es bei der Überarbeitung des Projektantrags und bei der Erstellung der Projektplanung. Die Überarbeitung des Projektantrags ging schneller vonstatten als gedacht, da erwartet wurde, dass mehr überarbeitet werden muss. Bei der Projektplanung war genau das Gegenteil der Fall, es wurde unterschätzt, wie lange man benötigt, um die Userstories herzustellen. \newline
\begin{table}
\begin{tabularx}{\textwidth}{X|X|X|X|X}
    \textbf{Titel} & \textbf{Soll} & \textbf{Ist} & \textbf{Differenz} & \textbf{Kommentar} \\ \hline
    Überarbeitung Projektantrag & 3h & 2h 30min & 30min & Weniger zu Überarbeiten als erwartet \\ \hline
    Projektplanung erstellen & 6h & 7h & 1h & Userstories schreiben und in PDF zu konvertieren hat viel Zeit gekostet \\ \hline
    Java und Android Studio lernen & 5h & 12h 30min & 7h 30min & Grosse Motivation zum Java lernen
\end{tabularx}
\caption{Tabelle erstes Zwischen-Controlling}
\end{table}
\newpage

\section{Zweites Zwischen-Controlling: Produktvorarbeit \\(14.06.2024)}
Das Projekt ist, bis jetzt, ziemlich wie erwartet verlaufen. Es wurde erwartet, dass mehr Zeit für das Lernen von Java und Android Studio verwendet wird, aufgrund des letzten Controllings. Der Prototyp hatte den erwarteten Aufwand. Der Bericht wurde stark unterschätzt. Grund für die Überzeit war die schlechte Kommunikation mit dem Coach welche zu einem Wissensmangel führte. Die Kommunikation soll bis zur nächsten Abgabe verbessert werden, um dieses Problem zu lösen. GitHub einzurichten hat auch unerwartet viel Zeit gekostet, die nicht eingeplant war. GitHub funktioniert nun einwandfrei und wird keine Zeit mehr benötigen. Die Gesamtzeit für diese Abgabe liegt bei 40h 15min und ist somit 15min über der erwarteten Zeit, was in Ordnung ist.
\begin{table}[h]
    \begin{tabularx}{\textwidth}{X|X|X|X|X}
        \textbf{Titel} & \textbf{Soll} & \textbf{Ist} & \textbf{Differenz} & \textbf{Kommentar} \\ \hline
        Prototyp erstellen & 10h & 10h & 0h & War der erwartete Aufwand \\\hline
        Java und Android Studio lernen & 15h & 9h 45min & 5h 15min & Das erste mal viel zu viel gemacht deshalb wurde das diesmal auch erwartet \\\hline
        Projektvorarbeit erstellen & 15h & 19h &  4h & Aufwand wurde unterschätzt \\\hline
        Unerwartetes & 0h & 1h 30min & 1h 30min & GitHub zu beginn noch nicht geplannt \\
    \end{tabularx}
    \caption{Tabelle zweites Zwischen-Controlling}
\end{table}

\newpage
\section{Drittes Zwischen-Controlling: Produkt und Reflexion (7.10.2024)}

Das Projekt ist ungefähr wie erwartet verlaufen. Um alle Userstories umzusetzten wäre ein Zeitaufwand von circa 70h geplant gewesen. Für alle projektzielrelevanten Userstories wären nur 28h geplant gewesen. Ich habe 40h 40min benötigt, um 20 der Userstories zu machen. Die 20 Userstories umfassen alle projektzielrelevanten und die beiden Userstories zu den Videos für die Knoten und für die Verbände. Diese 20 Userstories haben eine geplante Zeit von 33h. Die sieben Stunden Differenz ist hierbei vor allem auf die Tatsache zurückzuführen, dass in dieser Zeit das Erstellen der Themenübersichten und des Registers nicht eingeplant war, dies dauerte ca. 5.5h. Die restlichen 1.5h gingen zu lasten der Veröffentlichung im Google Play Store. Die Zeit, um den Bericht zu schreiben wurde genau eingehalten, so wurden 15h erwartet und das Ganze dauerte auch 15h. Die übrigen 4h 20min sind die Erstellung und Durchführung der Stichprobe und des Testlaufs. Diese beiden Punkte wurde im Zeitplan vergessen. Somit hatte diese Abgabe einen Aufwand von 60h, dieser liegt unter den erwarteten 85h. Wenn man für die Programmierung die 33h für die effektiv gemachten Userstories verwendet, statt der 70h aller Userstories, kommt man auf 48 erwartete Stunden.\par 
Das Produkt ist zum aktuellen Zeitpunkt so weit fertig, wie erwartet. Das Produkt hat die Funktionen, die es benötigt, um das Projektziel zu erfüllen, allerdings fehlen ihm die Funktionen der 18 nicht gemachten Userstories. Dies ist zwar schade, aber es war von Anfang an geplant nicht alle Userstories bis zur Abgabe zu machen, da es sich bei einigen Funktionen um Wunschfunktionen handelt. \par 
Bei dieser Abgabe gab es das erste Mal Ausgaben. Es wurden 25Fr. für den Zugriff auf die Google Play Console bezahlt, um die App veröffentlichen zu können. 12.90Fr. wurde für Seile ausgegeben, diese Ausgabe war unerwartet, da ich den Testpersonen die Seile als Dank geschenkt habe. Die Seile wurden den Testpersonen zuerst zur Verfügung gestellt, um die Knoten besser üben zu können.

\begin{table}[h]
    \begin{tabularx}{\textwidth}{X|X|X|X|X}
        \textbf{Titel} & \textbf{Soll} & \textbf{Ist} & \textbf{Differenz} & \textbf{Kommentar} \\ \hline
        App programmieren & 70h & 40h 40min & 29h 20min & Weniger Programmiert als geplant \\\hline
        Testlauf und Stichprobe & 0h & 4h 20min & 4h 20min & Wurde in der Planung vergessen \\\hline
        Bericht schreiben & 15h & 15h & 0h & War der erwartete Aufwand \\
    \end{tabularx}
    \caption{Tabelle drittes Zwischen-Controlling}
\end{table}

\newpage
\section{Abschluss-Controlling(7.10.2024)}

Bei der urprünglichen Zeitplanung wurde die benötigte Zeit auf 150h geschätzt. Eine genaue Zusammenrechnung der einzelnen erwarteten Zeiten ergibt eine Zeit von 139h. Die effektiv benötigte Zeit liegt bei 122h und 15min. Dieser Unterschied kann überwiegend mit dem dritten Controlling begründet werden, dort wurde die Zeit, die es braucht, um die App zu programmieren um 29h 20min überschätzt. Wenn man diese Zeit von den 139h abzieht liegt man allerdings unter der geplanten Zeit. Dies kann vorwiegend auf drei Ursachen zurückgeführt werden. Erstens die grosse Motivation Java zu lernen, dort gab es einen Zeitüberschuss von 7h 30min, diese wurde allerdings bei der nächsten Abgabe wieder um 5h 15min reduziert, durch das Defizit im Bereich des Programmieren lernen. Zweitens das Erstellen des Berichts zur Projektvorarbeit, dieser dauerte 4h länger als geplant. Drittens war der Testlauf und die Stichprobe, welche nirgends in Zeitplanung einflossen und somit 4h 20min länger gedauert haben als geplant. \par
Die Ausgaben waren 37.90Fr. Hierbei waren 25Fr. für den Google Play Store und die 12.90Fr. für die Seile der Testlaufteilnehmer.

\begin{table}[h]
    \begin{tabularx}{\textwidth}{X|X|X|X|X}
        \textbf{Titel} & \textbf{Soll} & \textbf{Ist} & \textbf{Differenz} & \textbf{Kommentar} \\ \hline
        Erstes Zwischen-Controlling & 14h & 22h & 8h & Zuviel Programmieren gelernt \\\hline
        Zweite Zwischen-Controlling & 40h & 40h 15min & 15min & Wie erwartet \\\hline
        Drittes Zwischen-Controlling & 85h & 60h & 25h & Weniger programmiert als geplant und Testlauf vergessen \\\hline
        Total & 139h & 122h 15min & 16h 45min & \\
    \end{tabularx}
    \caption{Zeittabelle Abschluss-Controlling}
\end{table}

\begin{table}[h]
    \begin{tabularx}{\textwidth}{X|X|X|X|X}
        \textbf{Titel} & \textbf{Soll} & \textbf{Ist} & \textbf{Differenz} & \textbf{Kommentar} \\ \hline
        Google Play Store & 25Fr. & 25Fr. & 0Fr. & Wie erwartet \\\hline
        Unerwartetes & 50Fr. & 12.90Fr. & 37.10Fr. & Kaum unerwartete Kosten \\\hline
        Literatur & 20Fr. & 0Fr. & 20Fr. & Keine Literatur benötigt \\\hline
        Total & 95Fr. & 37.90Fr. & 57.10Fr. & \\
    \end{tabularx}
    \caption{Ausgabentabelle Abschluss-Controlling}
\end{table}