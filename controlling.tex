\setcounter{chapter}{4}
\chapter{Controlling}
\section{Erstes Zwischen-Controlling: Projektantrag und Projektplanung (07.04.2024)}

Das Projekt ist bis jetzt wie erwartet verlaufen, die einzige wirkliche Ausnahme sind die Stunden die investiert wurden um Java und Android Studio zu lernen. Hierbei war nämlich gerade anfangs die Motivation voll da, mit der Zeit hat man dann aber gemerkt, dass es noch zu früh ist um wirklich zu lernen Programmieren, deshalb wurden die Zeiten die ins Programmieren flossen reduziert. Kleinere Abweichungen gab es allerdings auch bei dem Überarbeiten des Projektantrags und bei der erstellung der Projektplannung. Die Überarbeitung des Projektantrags ging schneller vonstatten als gedacht, dies vorallem deshalb, da erwartet wurde, dass mehr überarbeitet werden muss. Bei der Projektplanung war genau das gegenteil der Fall es wurde unterschätzt wie lange man benötigt um die Userstories herzustellen. \newline \newline \newline
\begin{tabular}{p{0.25\textwidth}|p{0.10\textwidth}|p{0.10\textwidth}|p{0.14\textwidth}|p{0.25\textwidth}}
    \textbf{Titel} & \textbf{Soll} & \textbf{Ist} & \textbf{Differenz} & \textbf{Kommentar} \\ \hline
    Überarbeitung Projektantrag & 3h & 2h 30min & 30min & Weniger zu Überarbeiten als erwartet \\ \hline
    Projektplanung erstellen & 6h & 7h & 1h & Userstories schreiben und in PDF zu konvertieren hat viel Zeit gekostet \\ \hline
    Java und Android Studio lernen & 5h & 12h 30min & 7h 30min & Grosse Motivation zum Java lernen
\end{tabular}