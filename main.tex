\documentclass{report}
\usepackage{graphicx} % Required for inserting images
\usepackage[a4paper]{geometry}
\usepackage[acronym]{glossaries-extra}
\usepackage[Glenn]{fncychap}
\usepackage[german]{babel}
\usepackage[style=numeric-comp, backend=biber, sorting = none]{biblatex}
\usepackage{hyperref, cleveref, fancyhdr, tikz, microtype}
\usepackage{eso-pic, footmisc}
\usepackage{preamble}
\begin{document}
\AddToShipoutPicture*{\BackgroundPic}

\maketitle
\tableofcontents
\chapter{Projektantrag}
\section{Aussgangssituation}
Ich bin ein aktiver Leiter in der Pfadi Rhätikon Schiers. Allerdings haben wir in der Pfadi das Problem, dass unsere Teilnehmer*innen sich nicht genügend für die Etappen\footnote{Jährliche Prüfung in den Bereichen der Pfaditechnik} vorbereiten können. Dies liegt daran, dass die Eltern und auch Kinder keinen wirklichen Zugang zur Pfaditechnik haben. Um das zu ändern, habe ich vor, in meiner \gls{ma} eine App zu programmieren, welche es den Eltern und Teilnehmer*innen mit Handy, neben anderen Hilfsmitteln, ermöglicht, sich zu Hause auf die Etappen vorzubereiten. Es gibt drei Etappen mit steigendem Schwierigkeitsgrad. Meine App soll sich auf die erste Etappe konzentrieren und den Nutzer*innen ermöglichen sich auf diese in einem Selbststudium vorzubereiten\cite{anforderungen}. \par
Dieses Problem will ich, wie gesagt, mit einer App angehen, da ich persönlich schon lange einmal eine solche programmieren wollte. Ein weiterer Grund ist, dass man mit einer App die Möglichkeit hat, auf Sensoren des Handys zuzugreifen. So wäre es zum Beispiel möglich, einen funktionierenden Kompass in die App zu integrieren, für alle die keinen zu Hause haben. Ich habe das Gefühl, dass es den Nutzern deutlich angenehmer ist, auf eine App zuzugreifen, da man nicht immer über den Browser gehen muss oder sein Büchlein suchen, um auf die Daten zuzugreifen. Die App-Programmierung will ich mit der Desktop-App Android Studio und der Programmiersprache Java angehen. Dies aus dem Grund, weil es so vom Buch \glqq Informatik für Dummies"\ vorgeschlagen wird \cite{dummies}. Ausserdem ist Android das am meisten verwendete Betriebssystem für Smartphones, weshalb ich mich dafür entschieden habe \cite{statistik_android}.
\par
Ein weiterer Punkt ist, dass ich die Arbeit in \LaTeX \ verfassen möchte. Ich habe mich dafür entschieden, weil es mir die Formatierung erleichtert und ich so zum Beispiel weniger Aufwand beim Erstellen des Inhaltsverzeichnis habe oder auch beim Zitieren von Quellen.

\section{Projektauftrag}
Mein Projektauftrag lautet wie folgt: \textbf{Entwickle und veröffentliche eine Android App, auf Java programmiert, die es Eltern und Pfadis\footnote{Pfaditeilnehmer} mit Handy ermöglicht, sich selbst die Pfaditechniken der ersten Etappe beizubringen.}

\section{Projektziele und Lernziele}
\subsection*{Projektziele}
\begin{enumerate}
    \item Entwicklung einer App, die alle Bereiche der ersten Etappe der Pfadi Rhätikon Schiers abdeckt, bis zum Testlauf.
    \item Durchführung eines Testlaufs mit 3 Personen ohne Pfadierfahrung, welche im Selbststudium mit der App, sich eigenständig auf eine erst Etappenprüfung verbereiten und sich dieser Prüfung stellen. Testlauf findet vor der Durchführung der Stichproben,
    \item Die App soll mit einer Stichprobe einer Zielgruppe von 5 Pfadis und 5 Eltern getestet werden und somit Feedback gesammelt werden, um so einen Überblick über die Qualität der App zu haben, bis 5 Wochen vor der Abgabe der \gls{ma}.
\end{enumerate}
\subsection*{Lernziele}
\begin{enumerate}
    \item Erlernen der Programmiersprache Java, um bis zur Durchführung des Testlaufs eine funktionsfähige App programmieren zu können.
    \item Vertiefen der Kenntnisse in Android Studio, um eine funktionsfähige Android-App mit Android Studio zu entwickeln und im Google Play zu veröffentlichen. Bis zur Durchführung des Testlaufs.
    \item Verbesserung der LaTeX-Kenntnisse, um die \gls{ma} in \LaTeX zu formatieren und zu zitieren, bis zum Beginn der Sommerferien.
\end{enumerate}

\section{Vorgehen}
Um dieses Projekt durchführen zu können, müssen zuerst die nötigen Java und Android Studio Fähigkeiten angeeignet werden. Das Vorgehen bis zur 3. Abgabe der \gls{ma} wird deshalb hauptsächlich daraus bestehen, sich die nötigen Fähigkeiten in Java und Android Studio anzueignen. Paralell dazu wird natürlich noch der Bericht für die 3. Abgabe geschrieben, welcher die Fähigkeiten bezüglich \LaTeX auf das gewünschte Level heben sollte. \par
Nach der 3. Abgabe geht es an das Programmieren der App. Das Programmieren wird mithilfe von Userstories geplant. Jede Userstory wird eine kleine Aufgabe, die benötigte Zeit, den Nutzen und die Priorität enthalten. Dann werden die Userstories nach und nach abgearbeitet beginnend bei der mit der höchsten Priorität. \par
Sobald die App in ihrem finalen Zustand ist wird sie dann noch auf dem Google Play Store hochgeladen, um sie der Öffentlichkeit zugänglich zu machen.

\chapter{Projektplanung}

\section{Projektorganisation}
\begin{center}
\begin{tabular}{c|c}
    \textbf{Person} & \textbf{Funktion} \\ \hline
    Andrea Bardill & Projektleiter \\
    Michael Brand & Auftraggeber der Schule \\
    5 Eltern \& 5 Teilnehmer*innen & Teilnehmer*innen der Stichprobe \\
    3 Personen ohne Pfadierfahrung & Teilnehmer*innen der Testung
\end{tabular}
\end{center}

\section{Projektstruktur und Zeitplan}
Um das ganze Projekt zu strukturieren wurde die Mehtode der Userstories gewählt \cite{userstories}. Userstories erlauben es alles was man ereichen möchte in kleine Aufgaben herunter zu brechen, um diese dann schön systematisch abzuarbeiten. Ein weiterer Vorteil dieser Userstories ist es, dass man wenn man zu wenig Zeit hat um alle zu beenden, denoch ein tolles Produkt hat da das wichtigste am Anfang erledigt wird. \par Den Aufgaben werden dann im Anschluss je eine Zeit zugeteilt, die es etwa benötigt um die Aufgabe zu bewältigen. Aus der Zusammenrechnung all dieser Zeiten ging hervor, dass dieses Projekt circa 150 Stunden in anspruch nehmen wird. Die Userstories wurden in einem externen Dokument abgegeben worden.

\section{Ressourcen und Finanzierung}

\begin{center}
\begin{tabular}{c|c|c}
    \textbf{Ressource} & \textbf{Betrag für Ressource (Fr.)} & \textbf{Finanzierung} \\ \hline
    Laptop & Vorhanden & - \\
    Intellij IDEA & 0 & - \\
    Android Studio & 0 & - \\
    \LaTeX -Editor & 0 & - \\
    Aufladen im Google Play Store & 25 & Privat \\
    Unerwartetes & 50 & Privat \\ \hline
    \textbf{Total} & 75 Fr. & 
\end{tabular}
\end{center}

\section{Projektumfeldanalyse}
\begin{tabular}{p{0.25\textwidth}|p{0.5\textwidth}|p{0.25\textwidth}}
    \textbf{Einflussfaktor} & \textbf{Auswirkungen/Folgen} & \textbf{Massnahmen} \\ \hline
    Laptop geht kaputt & Daten gehen verloren, Projekt kann nicht rechtzeitig fertiggestellt werden & Alle wichtigen Dateien sind auf einer Cloud gespeichert \\ \hline
    Zu wenige Personen für die Stichprobe und den Testlauf & Ungenaue Informationen aus dem Testlauf und der Stichprobe & Früh damit beginnen Personen zu sammeln \\ \hline
    Personen des Testlaufs trainieren nicht mit der App & Die Personen haben nicht das erwartete wissen, da zu wenig gelernt wurde & Die Testpersonen müssen aufschreiben wieviel sie mit der App gelernt haben \\
\end{tabular}

\setcounter{chapter}{4}
\chapter{Controlling}
\section{Erstes Zwischen-Controlling: Projektantrag und Projektplanung (07.04.2024)}

Das Projekt ist bis jetzt wie erwartet verlaufen, die einzige wirkliche Ausnahme sind die Stunden die investiert wurden um Java und Android Studio zu lernen. Hierbei war nämlich gerade anfangs die Motivation voll da, mit der Zeit hat man dann aber gemerkt, dass es noch zu früh ist um wirklich zu lernen Programmieren, deshalb wurden die Zeiten die ins Programmieren flossen reduziert. Kleinere Abweichungen gab es allerdings auch bei dem Überarbeiten des Projektantrags und bei der erstellung der Projektplannung. Die Überarbeitung des Projektantrags ging schneller vonstatten als gedacht, dies vorallem deshalb, da erwartet wurde, dass mehr überarbeitet werden muss. Bei der Projektplanung war genau das gegenteil der Fall es wurde unterschätzt wie lange man benötigt um die Userstories herzustellen. \newline \newline \newline
\begin{tabular}{p{0.25\textwidth}|p{0.10\textwidth}|p{0.10\textwidth}|p{0.14\textwidth}|p{0.25\textwidth}}
    \textbf{Titel} & \textbf{Soll} & \textbf{Ist} & \textbf{Differenz} & \textbf{Kommentar} \\ \hline
    Überarbeitung Projektantrag & 3h & 2h 30min & 30min & Weniger zu Überarbeiten als erwartet \\ \hline
    Projektplanung erstellen & 6h & 7h & 1h & Userstories schreiben und in PDF zu konvertieren hat viel Zeit gekostet \\ \hline
    Java und Android Studio lernen & 5h & 12h 30min & 7h 30min & Grosse Motivation zum Java lernen
\end{tabular}

\printbibliography
\end{document}