\chapter*{Vorwort}

Diese Arbeit wurde im Rahmen der Maturitätsarbeit an der EMS geschrieben. Ich habe dieses Thema gewählt, da ich seit einigen Jahren Pfadileiter bin und mir in dieser Zeit öfters das Problem aufgefallen ist, dass Pfadfinder, die keine Pfadfinder als Eltern haben, einen erschwerten Zugriff zur Pfaditechnik haben. Um dieses Problem zu bewältigen, habe ich mich entschieden, als Maturitätsarbeit, eine App zu programmieren. Gleichzeitig konnte ich mir meinen Wunsch eine App zu programmieren erfüllen. \par
Danken möchte ich Michael Brand, welcher mein Coach für diese Arbeit ist und welcher mir immer geholfen hat, wenn eine Frage oder ein Problem auftauchte. Weiter möchte ich meinem Beisitzer Mirco Auer danken, welcher ohne zu zögern zugesagt hat und sich die Zeit nahm, sich mit meiner Arbeit auseinander zu setzen. Ein weiteres Dankeschön geht an die EMS, welche mir das Umfeld für diese Arbeit geliefert hat und sowohl den Coach, als auch den Beisitzer zur Verfügung gestellt hat. Des Weiteren möchte ich noch allen Testpersonen dafür danken, dass sie sich freiwillig die Mühe gemacht haben, sich eine Woche lang mit der App auseinander zu setzen. Ein weiteres Dankeschön geht an die Eltern der Pfadfinder, welche bei den Stichproben mitgemacht haben und mir somit wichtiges Feedback gegeben haben. Ein letztes Dankeschön geht an meine Eltern, welche mich während der ganzen Arbeit unterstützt haben und kleine Dinge wie zum Beispiel, den Job der Kamerafrau für die Videos übernommen haben. 
\vfill
\noindent
Conters, 7.10.2024, Andrea Bardill