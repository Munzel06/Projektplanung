\chapter*{Abstract}

In dieser Arbeit wurde eine Android App namens Pfaditechnik programmiert. Das Ziel dieser App ist es Pfadfindern und Eltern von Pfadfindern zu ermöglichen sich selbstständig auf die erste Etappe der Pfadi Rhätikon Schiers vorzubereiten. Um sicherzustellen, dass dieses Ziel auch erfüllt ist, wurden sowohl ein Testlauf als auch eine Stichprobe durchgeführt. Das Ziel des Testlaufs war es, dass drei Personen ohne Pfadierfahrung sich für eine Woche mit der App auseinandersetzen, um zu zeigen, ob die App ihre Funktion erfüllt. Kurze Zeit nach diesem Testlauf wurde die Stichprobe durchgeführt. Das Ziel der Stichprobe war es, Feedback zur App aus der Zielgruppe zu bekommen, um sie danach anhand von diesem Feedback zu verbessern. \par
Für die Umsetzung dieses Projekts wurde damit begonnen einen Plan zu erstellen, was alles in der App vorhanden sein soll. All diese Elemente wurden dann in Userstories niedergeschrieben, welche den Arbeitsplan dieser Arbeit darstellten.\par
Nach der Fertigstellung des Plans wurden die nötigen Informationen gesammelt. Dies beinhaltete sowohl die einzelnen Bereiche der Pfaditechnik, als auch das Erlernen der notwendigen Kenntnisse der Programmiersprache Java.\par
Der nächste Schritt war die Programmierung der App. Die App wurde in der Desktop App Android Studio programmiert, in der Programmiersprache Java. In der App werden die Informationen zur Pfaditechnik in Text, Bild und Video dargestellt, wobei die Videos extra für dieses Projekt gedreht wurden. Nach der Fertigstellung der App wurde der Testlauf und die Stichprobe durchgeführt, welche beide zeigten, dass die App ihre Funktion erfüllt.\par
Die fertige App beinhaltet einen Homescreen mit Register, sieben Themenseiten und 16 Informationsseiten, in denen man sich das Wissen für die erste Etappe aneignen kann. Die App wird voraussichtlich ab dem 16.12.2024 im Google Play Store erhältlich sein.