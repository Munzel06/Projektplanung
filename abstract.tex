\chapter*{Abstract}

Diese Arbeit zeigt die Entwicklung einer Android App namens Pfaditechnik auf. Das Ziel dieser App ist es Pfadfinder*innen und ihren Eltern zu ermöglichen, sich selbstständig auf die erste Etappe der Pfadi Rhätikon Schiers vorzubereiten. Das Endprodukt ist eine App, welche die sechs Pfaditechniken Pionier, Samariter, Übermittlung, Karte und Kompass, Natur und Pfadigeschichte umfasst. Die einzelnen Techniken beinhalten jeweils eine Themenseite mit verschiedenen Informationsunterseiten. Die Informationen sind in Text, Bild und Video dargestellt. \par
Die Entwicklung der App begann mit der Niederschrift der Ziele in Userstories. Die erforderten Programmierkenntnisse wurden erlernt und mit diesen die App programmiert. \par
Zur Zielüberprüfung wurden ein Testlauf und eine Stichprobe durchgeführt. Im Testlauf haben vier Personen ohne Pfadierfahrung sich für eine Woche mit der App auseinandergesetzt, um zu testen, ob die App ihre Funktion erfüllt. Die Stichprobe holte Feedback zur App aus der realen Zielgruppe ein und zeigte Verbesserungspotential auf. \par
Die App wurde mit dem Feedback des Testlaufs und der Stichprobe weiterentwickelt und befindet sich nun im Stadium der Veröffentlichung. Langfristig soll sie über den Google Play Store verfügbar sein und Pfadis beim Erlernen der Pfaditechniken unterstützen.